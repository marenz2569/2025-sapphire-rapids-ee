\documentclass[conference,table,xcdraw,10pt,final]{IEEEtran}
% Comment out for Camera-Ready, leave in during writing.
\newcommand{\DraftModeOn}[0]{}

\usepackage{microtype}
\usepackage{flushend}

\usepackage[utf8]{inputenc}

\usepackage[T1]{fontenc}
\usepackage{amsmath}
\usepackage{bm}
\usepackage{xspace}
\usepackage{booktabs}
\usepackage[binary-units=true]{siunitx}
\usepackage{bigfoot}

% For \url in BibTex
\usepackage{hyperref}

\usepackage{todonotes}

\usepackage{subfig}

\usepackage{dblfloatfix}

\usepackage{multirow}
\usepackage{pifont}

\usepackage{textcomp}

\usepackage{amsfonts}

% for \cref
%\usepackage{cleveref}

%\usepackage{listings}


% === INSERT CUSTOM COMMANDS ===

% Use \xspace at the end to get, for example:
% \newcommand{\lotus}{\texttt{lo2s}\xspace}


% === Title page stuff ===

\title{Energy Efficiency Features of the Intel Sapphire Rapids Architecture}

\author{
	\IEEEauthorblockN{Markus Schmidl¹}
	\IEEEauthorblockA{
		Center for Information Services and High Performance Computing (ZIH)\\
		Technische Universit\"{a}t Dresden, 01062 Dresden, Germany \\
		¹markus.schmidl@mailbox.tu-dresden.de
	}
}

% === Document start ===

\begin{document}
	
	% Unfortunately, this has to be inside of the document. Tex sucks.
	% === DON'T TOUCH TIS ===
	
	\newcommand{\mycite}[1]{\textit{``#1''}}
	
	\ifx\DraftModeOn\undefined
	
	% Please leave notes FOR a specific co-author by using these macros
	\newcommand{\todoms}[1]{}
	
	\newcommand{\figref}[1]{Figure~\ref{fig:#1}}
	\newcommand{\tabref}[1]{Table~\ref{tab:#1}}
	\newcommand{\secref}[1]{Section~\ref{sec:#1}}
	\newcommand{\lstref}[1]{Listing~\ref{lst:#1}}
	
	\newcommand{\papertarget}[2]{}
	
	\renewcommand{\todo}[1]{}
	
	\else
	
	% Please leave notes FOR a specific co-author by using these macros
	\newcommand{\todoms}[1]{\todo[color=orange!60,inline,size=\small]{Markus: #1}}
	
	\newcommand{\figref}[1]{\textcolor{red}{Figure~\ref{fig:#1}}}
	\newcommand{\tabref}[1]{\textcolor{red}{Table~\ref{tab:#1}}}
	\newcommand{\secref}[1]{\textcolor{red}{Section~\ref{sec:#1}}}
	\newcommand{\lstref}[1]{\textcolor{red}{Listing~\ref{lst:#1}}}
	
	\newcommand{\papertarget}[2]{\todo[inline]{Target: #1: \\Deadline: #2}}
	
	\fi
	
	\maketitle
	
	%\papertarget{EEHPC-PW}{18.06.2021}
	% === Start doing things ===
	
	\begin{abstract}
		Future abstract content.
	\end{abstract}
	\begin{IEEEkeywords}
	\end{IEEEkeywords}

	
	%
	%% === A todo note with Target and deadline is very helpful. ===
	%\papertarget{...}{...}
	%
	\todoms{Something for Markus}
	
	\section{Introduction}
	\label{sec:intro}
	
	\section{Background and Related Work on Energy Efficiency Mechanisms}
	\label{sec:background}
	
	\cite{Schoene_2019_SKL}

	\section{Architectural Details of “Sapphire Rapids” Processors}

	\section{Test System and Power Measurements}

	\subsection{Validating RAPL Accuracy}
	Since the Haswell microarchitecture Intel integrated current measurement 

	\subsection{RAPL Filters}

	\section{Power Management}
	``It will also be interesting to look at Sapphire Rapids, which also implements Golden Cove cores, including an analysis of user space idle states, and AVX frequencies.''

	\subsection{Infuence of Tiles}
	\todoms{Where is the PMU?}

	\subsection{Shared Frequency Domains}

	\subsection{Cdyn Classes}

	\subsection{Uncore Frequency Scaling}

	\subsection{Idle State Latencies}
	\todoms{Both user and normal C states}

	\section{Memory Hierarchy}
	\subsection{Cache Latencies and Bandwidth}

	\section{Conclusion and Future Work}
	\label{sec:summary}
	
	% Acknowledgments always without numbering
	\section*{Acknowledgments and Reproducibility}
	Measurement programs, raw data, and chart notebooks are available at \url{TODO}.
	
	% === BibTex style selection ===
	% the MyBst is a very short and concise format, use with care.
	% Only First author is shown followed by 'et. al.' is needed.
	% \bibliographystyle{MyBst}
	\bibliographystyle{IEEEtran}
	
	% include a Bibliography from a file called 'paper.bib'
	\bibliography{paper}
	
\end{document}