\TUDoption{abstract}{section, multiple}

\begin{abstract}[pagestyle=empty.tudheadings]
In High Performance Computing, computational througput is used as key system metric.
Energy efficiency is however an important metric for scaling out workloads.
The 4th Generation Intel Xeon Scalable Processor, codenamed Sapphire Rapids, includes many improvements and new hardware control loops used to tune energy efficiency.
Little information is publicly discloused leading to a lack of knowledge about the improvements.
Information about the internals of these mechanisms is however, a prerequisite to tune systems for improved energy efficiency.
This thesis highlights changes in internal control mechanisms.
Key findings include qualitative and quantitative descriptions of the core and uncore frequency transtions, improvements in turbo frequencies and their allocation to cores, an evaluation of the internal power measurements, changed sleep wakeup latencies, and a description of the core microarchitecture.
Moreover, I present a model to improve core to core communication latencies that result from the 2d-mesh design of the L3 cache.
\end{abstract}
