% \chapter{Appendix I}
% \label{sec:appendix}
% \blindtext[3]
%
% \chapter{Appendix II}
% \label{sec:appendix2}
% \blindtext[3]

\chapter{Appendix Core to Core Communication Latencies Measurements}
\label{app:core_to_core_latencies_complete}
This section includes all scatter plots for all measurements conducted with the core to core communication latency benchmark as described in~\secref{l3_cache_slicing}.

\begin{figure}[]
    \centering
    \includegraphics[width=\columnwidth]{fig/core-to-core-latency/core-to-core-heatmap-min-800-800.pdf}
    \caption{Minimal core to core communication latency for hati's socket 0 with \SI{0.8}{\GHz} core and \SI{0.8}{\GHz} uncore frequency.}
\end{figure}
\begin{figure}[]
    \centering
    \includegraphics[width=\columnwidth]{fig/core-to-core-latency/core-to-core-heatmap-avg-800-800.pdf}
    \caption{Average core to core communication latency for hati's socket 0 with \SI{0.8}{\GHz} core and \SI{0.8}{\GHz} uncore frequency.}
\end{figure}
\begin{figure}[]
    \centering
    \includegraphics[width=\columnwidth]{fig/core-to-core-latency/core-to-core-heatmap-max-800-800.pdf}
    \caption{Maximimal core to core communication latency for hati's socket 0 with \SI{0.8}{\GHz} core and \SI{0.8}{\GHz} uncore frequency.}
\end{figure}
\begin{figure}[]
    \centering
    \includegraphics[width=\columnwidth]{fig/core-to-core-latency/core-to-core-heatmap-diff-800-800.pdf}
    \caption{Difference in core to core communication latency for hati's socket 0 with \SI{0.8}{\GHz} core and \SI{0.8}{\GHz} uncore frequency.}
\end{figure}

\begin{figure}[]
    \centering
    \includegraphics[width=\columnwidth]{fig/core-to-core-latency/core-to-core-heatmap-min-800-2500.pdf}
    \caption{Minimal core to core communication latency for hati's socket 0 with \SI{0.8}{\GHz} core and \SI{2.5}{\GHz} uncore frequency.}
\end{figure}
\begin{figure}[]
    \centering
    \includegraphics[width=\columnwidth]{fig/core-to-core-latency/core-to-core-heatmap-avg-800-2500.pdf}
    \caption{Average core to core communication latency for hati's socket 0 with \SI{0.8}{\GHz} core and \SI{2.5}{\GHz} uncore frequency.}
\end{figure}
\begin{figure}[]
    \centering
    \includegraphics[width=\columnwidth]{fig/core-to-core-latency/core-to-core-heatmap-max-800-2500.pdf}
    \caption{Maximimal core to core communication latency for hati's socket 0 with \SI{0.8}{\GHz} core and \SI{2.5}{\GHz} uncore frequency.}
\end{figure}
\begin{figure}[]
    \centering
    \includegraphics[width=\columnwidth]{fig/core-to-core-latency/core-to-core-heatmap-diff-800-2500.pdf}
    \caption{Difference in core to core communication latency for hati's socket 0 with \SI{0.8}{\GHz} core and \SI{2.5}{\GHz} uncore frequency.}
\end{figure}

\begin{figure}[]
    \centering
    \includegraphics[width=\columnwidth]{fig/core-to-core-latency/core-to-core-heatmap-min-2000-800.pdf}
    \caption{Minimal core to core communication latency for hati's socket 0 with \SI{2.0}{\GHz} core and \SI{0.8}{\GHz} uncore frequency.}
\end{figure}
\begin{figure}[]
    \centering
    \includegraphics[width=\columnwidth]{fig/core-to-core-latency/core-to-core-heatmap-avg-2000-800.pdf}
    \caption{Average core to core communication latency for hati's socket 0 with \SI{2.0}{\GHz} core and \SI{0.8}{\GHz} uncore frequency.}
\end{figure}
\begin{figure}[]
    \centering
    \includegraphics[width=\columnwidth]{fig/core-to-core-latency/core-to-core-heatmap-max-2000-800.pdf}
    \caption{Maximimal core to core communication latency for hati's socket 0 with \SI{2.0}{\GHz} core and \SI{0.8}{\GHz} uncore frequency.}
\end{figure}
\begin{figure}[]
    \centering
    \includegraphics[width=\columnwidth]{fig/core-to-core-latency/core-to-core-heatmap-diff-2000-800.pdf}
    \caption{Difference in core to core communication latency for hati's socket 0 with \SI{2.0}{\GHz} core and \SI{0.8}{\GHz} uncore frequency.}
\end{figure}

\begin{figure}[]
    \centering
    \includegraphics[width=\columnwidth]{fig/core-to-core-latency/core-to-core-heatmap-min-2000-2500.pdf}
    \caption{Minimal core to core communication latency for hati's socket 0 with \SI{2.0}{\GHz} core and \SI{2.5}{\GHz} uncore frequency.}
\end{figure}
\begin{figure}[]
    \centering
    \includegraphics[width=\columnwidth]{fig/core-to-core-latency/core-to-core-heatmap-avg-2000-2500.pdf}
    \caption{Average core to core communication latency for hati's socket 0 with \SI{2.0}{\GHz} core and \SI{2.5}{\GHz} uncore frequency.}
\end{figure}
\begin{figure}[]
    \centering
    \includegraphics[width=\columnwidth]{fig/core-to-core-latency/core-to-core-heatmap-max-2000-2500.pdf}
    \caption{Maximimal core to core communication latency for hati's socket 0 with \SI{2.0}{\GHz} core and \SI{2.5}{\GHz} uncore frequency.}
\end{figure}
\begin{figure}[]
    \centering
    \includegraphics[width=\columnwidth]{fig/core-to-core-latency/core-to-core-heatmap-diff-2000-2500.pdf}
    \caption{Difference in core to core communication latency for hati's socket 0 with \SI{2.0}{\GHz} core and \SI{2.5}{\GHz} uncore frequency.}
\end{figure}

\begin{figure}[]
    \centering
    \includegraphics[width=\columnwidth]{fig/core-to-core-latency/core-to-core-heatmap-min-3800-800.pdf}
    \caption{Minimal core to core communication latency for hati's socket 0 with \SI{3.8}{\GHz} core and \SI{0.8}{\GHz} uncore frequency.}
\end{figure}
\begin{figure}[]
    \centering
    \includegraphics[width=\columnwidth]{fig/core-to-core-latency/core-to-core-heatmap-avg-3800-800.pdf}
    \caption{Average core to core communication latency for hati's socket 0 with \SI{3.8}{\GHz} core and \SI{0.8}{\GHz} uncore frequency.}
\end{figure}
\begin{figure}[]
    \centering
    \includegraphics[width=\columnwidth]{fig/core-to-core-latency/core-to-core-heatmap-max-3800-800.pdf}
    \caption{Maximimal core to core communication latency for hati's socket 0 with \SI{3.8}{\GHz} core and \SI{0.8}{\GHz} uncore frequency.}
\end{figure}
\begin{figure}[]
    \centering
    \includegraphics[width=\columnwidth]{fig/core-to-core-latency/core-to-core-heatmap-diff-3800-800.pdf}
    \caption{Difference in core to core communication latency for hati's socket 0 with \SI{3.8}{\GHz} core and \SI{0.8}{\GHz} uncore frequency.}
\end{figure}

\begin{figure}[]
    \centering
    \includegraphics[width=\columnwidth]{fig/core-to-core-latency/core-to-core-heatmap-min-3800-2500.pdf}
    \caption{Minimal core to core communication latency for hati's socket 0 with \SI{3.8}{\GHz} core and \SI{2.5}{\GHz} uncore frequency.}
\end{figure}
\begin{figure}[]
    \centering
    \includegraphics[width=\columnwidth]{fig/core-to-core-latency/core-to-core-heatmap-avg-3800-2500.pdf}
    \caption{Average core to core communication latency for hati's socket 0 with \SI{3.8}{\GHz} core and \SI{2.5}{\GHz} uncore frequency.}
\end{figure}
\begin{figure}[]
    \centering
    \includegraphics[width=\columnwidth]{fig/core-to-core-latency/core-to-core-heatmap-max-3800-2500.pdf}
    \caption{Maximimal core to core communication latency for hati's socket 0 with \SI{3.8}{\GHz} core and \SI{2.5}{\GHz} uncore frequency.}
\end{figure}
\begin{figure}[]
    \centering
    \includegraphics[width=\columnwidth]{fig/core-to-core-latency/core-to-core-heatmap-diff-3800-2500.pdf}
    \caption{Difference in core to core communication latency for hati's socket 0 with \SI{3.8}{\GHz} core and \SI{2.5}{\GHz} uncore frequency.}
\end{figure}

\chapter{Appendix Core to Core Communication Latencies Models}
\label{app:core_to_core_latencies_model}
This section includes the plots for model of the core to core communication latencies as described in~\secref{l3_cache_slicing}.

\begin{figure}[]
    \centering
    \includegraphics[width=\columnwidth]{fig/core-to-core-latency/all-to-all-heatmap-model-min.pdf}
    \caption{Model of the minimal core to core communication latency over all cores of hati's socket 0.}
\end{figure}

\begin{figure}[]
    \centering
    \includegraphics[width=\columnwidth]{fig/core-to-core-latency/all-to-all-heatmap-model-max.pdf}
    \caption{Model of the maximal core to core communication latency over all cores of hati's socket 0.}
\end{figure}

\begin{figure}[]
    \centering
    \includegraphics[width=\columnwidth]{fig/core-to-core-latency/all-to-all-heatmap-model-diff.pdf}
    \caption{Model of the difference in core to core communication latency over all cores of hati's socket 0.}
\end{figure}


\chapter{Appendix Validating RAPL Accuracy}
\label{app:validating_rapl_accuracy}
This section includes the plots for the correlation of the external reference measurent against individual RAPL metrics as described in~\secref{validating_rapl_accuracy}.

\begin{figure}[]
    \centering
    \includegraphics[width=0.8\columnwidth]{fig/rapl-accuracy/rapl-accuracy-package.pdf}
    \caption{The roco2 microbenchmark is executed on a varying number of cores with different frequencies.
    The RAPL package measurement is mapped with a quadratic fit to the external reference measurement for compute kernels which do no access the DRAM.}
\end{figure}

\begin{figure}[]
    \centering
    \includegraphics[width=0.8\columnwidth]{fig/rapl-accuracy/rapl-accuracy-dram.pdf}
    \caption{The roco2 microbenchmark is executed on a varying number of cores with different frequencies.}
\end{figure}

\chapter{Appendix RAPL Filters}
\label{app:rapl_filters}
This section includes the plots of the RAPL filter measurement for all core frequencies as described in~\secref{rapl_filters}.

\clearpage

\begin{figure}[]
    \centering
    % width could be 1, 0.9, 0.8 instead
    \includegraphics[width=\columnwidth]{fig/rapl-update-intervals/MSR_PACKAGE_ENERGY_TIME_STATUS_800000.pdf}
    \caption{Kernel density estimation of the time of measurement vs read energy value of the RAPL package counter of the first socket during for two constant power draw scenarios with and without filters enabled.
    The left plot uses the \textbf{rdtsc} function to get a timestamp for the RAPL measurement, the right uses the newly integrated timestamp in the \textbf{MSR\_PACKAGE\_ENERGY\_TIME\_STATUS} MSR.
    The core frequency is set to \SI{0.8}{\GHz}.}
\end{figure}

\begin{figure}[]
    \centering
    % width cloud be 0.54, 0.49, 0.44 instead
    \includegraphics[width=0.54\columnwidth]{fig/rapl-update-intervals/MSR_RAM_ENERGY_800000.pdf}
    \caption{Kernel density estimation of the time of measurement via timestamp counter and the read energy value of the RAPL dram counter of the first socket during a constant power draw scenario.
    The core frequency is set to \SI{0.8}{\GHz}.}
\end{figure}

\clearpage
\begin{figure}[]
    \centering
    % width could be 1, 0.9, 0.8 instead
    \includegraphics[width=\columnwidth]{fig/rapl-update-intervals/MSR_PACKAGE_ENERGY_TIME_STATUS_900000.pdf}
    \caption{Kernel density estimation of the time of measurement vs read energy value of the RAPL package counter of the first socket during for two constant power draw scenarios with and without filters enabled.
    The left plot uses the \textbf{rdtsc} function to get a timestamp for the RAPL measurement, the right uses the newly integrated timestamp in the \textbf{MSR\_PACKAGE\_ENERGY\_TIME\_STATUS} MSR.
    The core frequency is set to \SI{0.9}{\GHz}.}
\end{figure}

\begin{figure}[]
    \centering
    % width cloud be 0.54, 0.49, 0.44 instead
    \includegraphics[width=0.54\columnwidth]{fig/rapl-update-intervals/MSR_RAM_ENERGY_900000.pdf}
    \caption{Kernel density estimation of the time of measurement via timestamp counter and the read energy value of the RAPL dram counter of the first socket during a constant power draw scenario.
    The core frequency is set to \SI{0.9}{\GHz}.}
\end{figure}

\clearpage
\begin{figure}[]
    \centering
    % width could be 1, 0.9, 0.8 instead
    \includegraphics[width=\columnwidth]{fig/rapl-update-intervals/MSR_PACKAGE_ENERGY_TIME_STATUS_1000000.pdf}
    \caption{Kernel density estimation of the time of measurement vs read energy value of the RAPL package counter of the first socket during for two constant power draw scenarios with and without filters enabled.
    The left plot uses the \textbf{rdtsc} function to get a timestamp for the RAPL measurement, the right uses the newly integrated timestamp in the \textbf{MSR\_PACKAGE\_ENERGY\_TIME\_STATUS} MSR.
    The core frequency is set to \SI{1.0}{\GHz}.}
\end{figure}

\begin{figure}[]
    \centering
    % width cloud be 0.54, 0.49, 0.44 instead
    \includegraphics[width=0.54\columnwidth]{fig/rapl-update-intervals/MSR_RAM_ENERGY_1000000.pdf}
    \caption{Kernel density estimation of the time of measurement via timestamp counter and the read energy value of the RAPL dram counter of the first socket during a constant power draw scenario.
    The core frequency is set to \SI{1.0}{\GHz}.}
\end{figure}

\clearpage
\begin{figure}[]
    \centering
    % width could be 1, 0.9, 0.8 instead
    \includegraphics[width=\columnwidth]{fig/rapl-update-intervals/MSR_PACKAGE_ENERGY_TIME_STATUS_1100000.pdf}
    \caption{Kernel density estimation of the time of measurement vs read energy value of the RAPL package counter of the first socket during for two constant power draw scenarios with and without filters enabled.
    The left plot uses the \textbf{rdtsc} function to get a timestamp for the RAPL measurement, the right uses the newly integrated timestamp in the \textbf{MSR\_PACKAGE\_ENERGY\_TIME\_STATUS} MSR.
    The core frequency is set to \SI{1.1}{\GHz}.}
\end{figure}

\begin{figure}[]
    \centering
    % width cloud be 0.54, 0.49, 0.44 instead
    \includegraphics[width=0.54\columnwidth]{fig/rapl-update-intervals/MSR_RAM_ENERGY_1100000.pdf}
    \caption{Kernel density estimation of the time of measurement via timestamp counter and the read energy value of the RAPL dram counter of the first socket during a constant power draw scenario.
    The core frequency is set to \SI{1.1}{\GHz}.}
\end{figure}

\clearpage
\begin{figure}[]
    \centering
    % width could be 1, 0.9, 0.8 instead
    \includegraphics[width=\columnwidth]{fig/rapl-update-intervals/MSR_PACKAGE_ENERGY_TIME_STATUS_1200000.pdf}
    \caption{Kernel density estimation of the time of measurement vs read energy value of the RAPL package counter of the first socket during for two constant power draw scenarios with and without filters enabled.
    The left plot uses the \textbf{rdtsc} function to get a timestamp for the RAPL measurement, the right uses the newly integrated timestamp in the \textbf{MSR\_PACKAGE\_ENERGY\_TIME\_STATUS} MSR.
    The core frequency is set to \SI{1.2}{\GHz}.}
\end{figure}

\begin{figure}[]
    \centering
    % width cloud be 0.54, 0.49, 0.44 instead
    \includegraphics[width=0.54\columnwidth]{fig/rapl-update-intervals/MSR_RAM_ENERGY_1200000.pdf}
    \caption{Kernel density estimation of the time of measurement via timestamp counter and the read energy value of the RAPL dram counter of the first socket during a constant power draw scenario.
    The core frequency is set to \SI{1.2}{\GHz}.}
\end{figure}

\clearpage
\begin{figure}[]
    \centering
    % width could be 1, 0.9, 0.8 instead
    \includegraphics[width=\columnwidth]{fig/rapl-update-intervals/MSR_PACKAGE_ENERGY_TIME_STATUS_1300000.pdf}
    \caption{Kernel density estimation of the time of measurement vs read energy value of the RAPL package counter of the first socket during for two constant power draw scenarios with and without filters enabled.
    The left plot uses the \textbf{rdtsc} function to get a timestamp for the RAPL measurement, the right uses the newly integrated timestamp in the \textbf{MSR\_PACKAGE\_ENERGY\_TIME\_STATUS} MSR.
    The core frequency is set to \SI{1.3}{\GHz}.}
\end{figure}

\begin{figure}[]
    \centering
    % width cloud be 0.54, 0.49, 0.44 instead
    \includegraphics[width=0.54\columnwidth]{fig/rapl-update-intervals/MSR_RAM_ENERGY_1300000.pdf}
    \caption{Kernel density estimation of the time of measurement via timestamp counter and the read energy value of the RAPL dram counter of the first socket during a constant power draw scenario.
    The core frequency is set to \SI{1.3}{\GHz}.}
\end{figure}

\clearpage
\begin{figure}[]
    \centering
    % width could be 1, 0.9, 0.8 instead
    \includegraphics[width=\columnwidth]{fig/rapl-update-intervals/MSR_PACKAGE_ENERGY_TIME_STATUS_1400000.pdf}
    \caption{Kernel density estimation of the time of measurement vs read energy value of the RAPL package counter of the first socket during for two constant power draw scenarios with and without filters enabled.
    The left plot uses the \textbf{rdtsc} function to get a timestamp for the RAPL measurement, the right uses the newly integrated timestamp in the \textbf{MSR\_PACKAGE\_ENERGY\_TIME\_STATUS} MSR.
    The core frequency is set to \SI{1.4}{\GHz}.}
\end{figure}

\begin{figure}[]
    \centering
    % width cloud be 0.54, 0.49, 0.44 instead
    \includegraphics[width=0.54\columnwidth]{fig/rapl-update-intervals/MSR_RAM_ENERGY_1400000.pdf}
    \caption{Kernel density estimation of the time of measurement via timestamp counter and the read energy value of the RAPL dram counter of the first socket during a constant power draw scenario.
    The core frequency is set to \SI{1.4}{\GHz}.}
\end{figure}

\clearpage
\begin{figure}[]
    \centering
    % width could be 1, 0.9, 0.8 instead
    \includegraphics[width=\columnwidth]{fig/rapl-update-intervals/MSR_PACKAGE_ENERGY_TIME_STATUS_1500000.pdf}
    \caption{Kernel density estimation of the time of measurement vs read energy value of the RAPL package counter of the first socket during for two constant power draw scenarios with and without filters enabled.
    The left plot uses the \textbf{rdtsc} function to get a timestamp for the RAPL measurement, the right uses the newly integrated timestamp in the \textbf{MSR\_PACKAGE\_ENERGY\_TIME\_STATUS} MSR.
    The core frequency is set to \SI{1.5}{\GHz}.}
\end{figure}

\begin{figure}[]
    \centering
    % width cloud be 0.54, 0.49, 0.44 instead
    \includegraphics[width=0.54\columnwidth]{fig/rapl-update-intervals/MSR_RAM_ENERGY_1500000.pdf}
    \caption{Kernel density estimation of the time of measurement via timestamp counter and the read energy value of the RAPL dram counter of the first socket during a constant power draw scenario.
    The core frequency is set to \SI{1.5}{\GHz}.}
\end{figure}

\clearpage
\begin{figure}[]
    \centering
    % width could be 1, 0.9, 0.8 instead
    \includegraphics[width=\columnwidth]{fig/rapl-update-intervals/MSR_PACKAGE_ENERGY_TIME_STATUS_1600000.pdf}
    \caption{Kernel density estimation of the time of measurement vs read energy value of the RAPL package counter of the first socket during for two constant power draw scenarios with and without filters enabled.
    The left plot uses the \textbf{rdtsc} function to get a timestamp for the RAPL measurement, the right uses the newly integrated timestamp in the \textbf{MSR\_PACKAGE\_ENERGY\_TIME\_STATUS} MSR.
    The core frequency is set to \SI{1.6}{\GHz}.}
\end{figure}

\begin{figure}[]
    \centering
    % width cloud be 0.54, 0.49, 0.44 instead
    \includegraphics[width=0.54\columnwidth]{fig/rapl-update-intervals/MSR_RAM_ENERGY_1600000.pdf}
    \caption{Kernel density estimation of the time of measurement via timestamp counter and the read energy value of the RAPL dram counter of the first socket during a constant power draw scenario.
    The core frequency is set to \SI{1.6}{\GHz}.}
\end{figure}

\clearpage
\begin{figure}[]
    \centering
    % width could be 1, 0.9, 0.8 instead
    \includegraphics[width=\columnwidth]{fig/rapl-update-intervals/MSR_PACKAGE_ENERGY_TIME_STATUS_1700000.pdf}
    \caption{Kernel density estimation of the time of measurement vs read energy value of the RAPL package counter of the first socket during for two constant power draw scenarios with and without filters enabled.
    The left plot uses the \textbf{rdtsc} function to get a timestamp for the RAPL measurement, the right uses the newly integrated timestamp in the \textbf{MSR\_PACKAGE\_ENERGY\_TIME\_STATUS} MSR.
    The core frequency is set to \SI{1.7}{\GHz}.}
\end{figure}

\begin{figure}[]
    \centering
    % width cloud be 0.54, 0.49, 0.44 instead
    \includegraphics[width=0.54\columnwidth]{fig/rapl-update-intervals/MSR_RAM_ENERGY_1700000.pdf}
    \caption{Kernel density estimation of the time of measurement via timestamp counter and the read energy value of the RAPL dram counter of the first socket during a constant power draw scenario.
    The core frequency is set to \SI{1.7}{\GHz}.}
\end{figure}

\clearpage
\begin{figure}[]
    \centering
    % width could be 1, 0.9, 0.8 instead
    \includegraphics[width=\columnwidth]{fig/rapl-update-intervals/MSR_PACKAGE_ENERGY_TIME_STATUS_1800000.pdf}
    \caption{Kernel density estimation of the time of measurement vs read energy value of the RAPL package counter of the first socket during for two constant power draw scenarios with and without filters enabled.
    The left plot uses the \textbf{rdtsc} function to get a timestamp for the RAPL measurement, the right uses the newly integrated timestamp in the \textbf{MSR\_PACKAGE\_ENERGY\_TIME\_STATUS} MSR.
    The core frequency is set to \SI{1.8}{\GHz}.}
\end{figure}

\begin{figure}[]
    \centering
    % width cloud be 0.54, 0.49, 0.44 instead
    \includegraphics[width=0.54\columnwidth]{fig/rapl-update-intervals/MSR_RAM_ENERGY_1800000.pdf}
    \caption{Kernel density estimation of the time of measurement via timestamp counter and the read energy value of the RAPL dram counter of the first socket during a constant power draw scenario.
    The core frequency is set to \SI{1.8}{\GHz}.}
\end{figure}

\clearpage
\begin{figure}[]
    \centering
    % width could be 1, 0.9, 0.8 instead
    \includegraphics[width=\columnwidth]{fig/rapl-update-intervals/MSR_PACKAGE_ENERGY_TIME_STATUS_1900000.pdf}
    \caption{Kernel density estimation of the time of measurement vs read energy value of the RAPL package counter of the first socket during for two constant power draw scenarios with and without filters enabled.
    The left plot uses the \textbf{rdtsc} function to get a timestamp for the RAPL measurement, the right uses the newly integrated timestamp in the \textbf{MSR\_PACKAGE\_ENERGY\_TIME\_STATUS} MSR.
    The core frequency is set to \SI{1.9}{\GHz}.}
\end{figure}

\begin{figure}[]
    \centering
    % width cloud be 0.54, 0.49, 0.44 instead
    \includegraphics[width=0.54\columnwidth]{fig/rapl-update-intervals/MSR_RAM_ENERGY_1900000.pdf}
    \caption{Kernel density estimation of the time of measurement via timestamp counter and the read energy value of the RAPL dram counter of the first socket during a constant power draw scenario.
    The core frequency is set to \SI{1.9}{\GHz}.}
\end{figure}

\clearpage
\begin{figure}[]
    \centering
    % width could be 1, 0.9, 0.8 instead
    \includegraphics[width=\columnwidth]{fig/rapl-update-intervals/MSR_PACKAGE_ENERGY_TIME_STATUS_2000000.pdf}
    \caption{Kernel density estimation of the time of measurement vs read energy value of the RAPL package counter of the first socket during for two constant power draw scenarios with and without filters enabled.
    The left plot uses the \textbf{rdtsc} function to get a timestamp for the RAPL measurement, the right uses the newly integrated timestamp in the \textbf{MSR\_PACKAGE\_ENERGY\_TIME\_STATUS} MSR.
    The core frequency is set to \SI{2.0}{\GHz}.}
\end{figure}

\begin{figure}[]
    \centering
    % width cloud be 0.54, 0.49, 0.44 instead
    \includegraphics[width=0.54\columnwidth]{fig/rapl-update-intervals/MSR_RAM_ENERGY_2000000.pdf}
    \caption{Kernel density estimation of the time of measurement via timestamp counter and the read energy value of the RAPL dram counter of the first socket during a constant power draw scenario.
    The core frequency is set to \SI{2.0}{\GHz}.}
\end{figure}

\clearpage
\begin{figure}[]
    \centering
    % width could be 1, 0.9, 0.8 instead
    \includegraphics[width=\columnwidth]{fig/rapl-update-intervals/MSR_PACKAGE_ENERGY_TIME_STATUS_2100000.pdf}
    \caption{Kernel density estimation of the time of measurement vs read energy value of the RAPL package counter of the first socket during for two constant power draw scenarios with and without filters enabled.
    The left plot uses the \textbf{rdtsc} function to get a timestamp for the RAPL measurement, the right uses the newly integrated timestamp in the \textbf{MSR\_PACKAGE\_ENERGY\_TIME\_STATUS} MSR.
    The core frequency is set to \SI{2.1}{\GHz}.}
\end{figure}

\begin{figure}[]
    \centering
    % width cloud be 0.54, 0.49, 0.44 instead
    \includegraphics[width=0.54\columnwidth]{fig/rapl-update-intervals/MSR_RAM_ENERGY_2100000.pdf}
    \caption{Kernel density estimation of the time of measurement via timestamp counter and the read energy value of the RAPL dram counter of the first socket during a constant power draw scenario.
    The core frequency is set to \SI{2.1}{\GHz}.}
\end{figure}

\clearpage
\begin{figure}[]
    \centering
    % width could be 1, 0.9, 0.8 instead
    \includegraphics[width=\columnwidth]{fig/rapl-update-intervals/MSR_PACKAGE_ENERGY_TIME_STATUS_2200000.pdf}
    \caption{Kernel density estimation of the time of measurement vs read energy value of the RAPL package counter of the first socket during for two constant power draw scenarios with and without filters enabled.
    The left plot uses the \textbf{rdtsc} function to get a timestamp for the RAPL measurement, the right uses the newly integrated timestamp in the \textbf{MSR\_PACKAGE\_ENERGY\_TIME\_STATUS} MSR.
    The core frequency is set to \SI{2.2}{\GHz}.}
\end{figure}

\begin{figure}[]
    \centering
    % width cloud be 0.54, 0.49, 0.44 instead
    \includegraphics[width=0.54\columnwidth]{fig/rapl-update-intervals/MSR_RAM_ENERGY_2200000.pdf}
    \caption{Kernel density estimation of the time of measurement via timestamp counter and the read energy value of the RAPL dram counter of the first socket during a constant power draw scenario.
    The core frequency is set to \SI{2.2}{\GHz}.}
\end{figure}

\clearpage
\begin{figure}[]
    \centering
    % width could be 1, 0.9, 0.8 instead
    \includegraphics[width=\columnwidth]{fig/rapl-update-intervals/MSR_PACKAGE_ENERGY_TIME_STATUS_2300000.pdf}
    \caption{Kernel density estimation of the time of measurement vs read energy value of the RAPL package counter of the first socket during for two constant power draw scenarios with and without filters enabled.
    The left plot uses the \textbf{rdtsc} function to get a timestamp for the RAPL measurement, the right uses the newly integrated timestamp in the \textbf{MSR\_PACKAGE\_ENERGY\_TIME\_STATUS} MSR.
    The core frequency is set to \SI{2.3}{\GHz}.}
\end{figure}

\begin{figure}[]
    \centering
    % width cloud be 0.54, 0.49, 0.44 instead
    \includegraphics[width=0.54\columnwidth]{fig/rapl-update-intervals/MSR_RAM_ENERGY_2300000.pdf}
    \caption{Kernel density estimation of the time of measurement via timestamp counter and the read energy value of the RAPL dram counter of the first socket during a constant power draw scenario.
    The core frequency is set to \SI{2.3}{\GHz}.}
\end{figure}

\clearpage
\begin{figure}[]
    \centering
    % width could be 1, 0.9, 0.8 instead
    \includegraphics[width=\columnwidth]{fig/rapl-update-intervals/MSR_PACKAGE_ENERGY_TIME_STATUS_2400000.pdf}
    \caption{Kernel density estimation of the time of measurement vs read energy value of the RAPL package counter of the first socket during for two constant power draw scenarios with and without filters enabled.
    The left plot uses the \textbf{rdtsc} function to get a timestamp for the RAPL measurement, the right uses the newly integrated timestamp in the \textbf{MSR\_PACKAGE\_ENERGY\_TIME\_STATUS} MSR.
    The core frequency is set to \SI{2.4}{\GHz}.}
\end{figure}

\begin{figure}[]
    \centering
    % width cloud be 0.54, 0.49, 0.44 instead
    \includegraphics[width=0.54\columnwidth]{fig/rapl-update-intervals/MSR_RAM_ENERGY_2400000.pdf}
    \caption{Kernel density estimation of the time of measurement via timestamp counter and the read energy value of the RAPL dram counter of the first socket during a constant power draw scenario.
    The core frequency is set to \SI{2.4}{\GHz}.}
\end{figure}

\clearpage
\begin{figure}[]
    \centering
    % width could be 1, 0.9, 0.8 instead
    \includegraphics[width=\columnwidth]{fig/rapl-update-intervals/MSR_PACKAGE_ENERGY_TIME_STATUS_2500000.pdf}
    \caption{Kernel density estimation of the time of measurement vs read energy value of the RAPL package counter of the first socket during for two constant power draw scenarios with and without filters enabled.
    The left plot uses the \textbf{rdtsc} function to get a timestamp for the RAPL measurement, the right uses the newly integrated timestamp in the \textbf{MSR\_PACKAGE\_ENERGY\_TIME\_STATUS} MSR.
    The core frequency is set to \SI{2.5}{\GHz}.}
\end{figure}

\begin{figure}[]
    \centering
    % width cloud be 0.54, 0.49, 0.44 instead
    \includegraphics[width=0.54\columnwidth]{fig/rapl-update-intervals/MSR_RAM_ENERGY_2500000.pdf}
    \caption{Kernel density estimation of the time of measurement via timestamp counter and the read energy value of the RAPL dram counter of the first socket during a constant power draw scenario.
    The core frequency is set to \SI{2.5}{\GHz}.}
\end{figure}

\clearpage
\begin{figure}[]
    \centering
    % width could be 1, 0.9, 0.8 instead
    \includegraphics[width=\columnwidth]{fig/rapl-update-intervals/MSR_PACKAGE_ENERGY_TIME_STATUS_2600000.pdf}
    \caption{Kernel density estimation of the time of measurement vs read energy value of the RAPL package counter of the first socket during for two constant power draw scenarios with and without filters enabled.
    The left plot uses the \textbf{rdtsc} function to get a timestamp for the RAPL measurement, the right uses the newly integrated timestamp in the \textbf{MSR\_PACKAGE\_ENERGY\_TIME\_STATUS} MSR.
    The core frequency is set to \SI{2.6}{\GHz}.}
\end{figure}

\begin{figure}[]
    \centering
    % width cloud be 0.54, 0.49, 0.44 instead
    \includegraphics[width=0.54\columnwidth]{fig/rapl-update-intervals/MSR_RAM_ENERGY_2600000.pdf}
    \caption{Kernel density estimation of the time of measurement via timestamp counter and the read energy value of the RAPL dram counter of the first socket during a constant power draw scenario.
    The core frequency is set to \SI{2.6}{\GHz}.}
\end{figure}

\clearpage
\begin{figure}[]
    \centering
    % width could be 1, 0.9, 0.8 instead
    \includegraphics[width=\columnwidth]{fig/rapl-update-intervals/MSR_PACKAGE_ENERGY_TIME_STATUS_2700000.pdf}
    \caption{Kernel density estimation of the time of measurement vs read energy value of the RAPL package counter of the first socket during for two constant power draw scenarios with and without filters enabled.
    The left plot uses the \textbf{rdtsc} function to get a timestamp for the RAPL measurement, the right uses the newly integrated timestamp in the \textbf{MSR\_PACKAGE\_ENERGY\_TIME\_STATUS} MSR.
    The core frequency is set to \SI{2.7}{\GHz}.}
\end{figure}

\begin{figure}[]
    \centering
    % width cloud be 0.54, 0.49, 0.44 instead
    \includegraphics[width=0.54\columnwidth]{fig/rapl-update-intervals/MSR_RAM_ENERGY_2700000.pdf}
    \caption{Kernel density estimation of the time of measurement via timestamp counter and the read energy value of the RAPL dram counter of the first socket during a constant power draw scenario.
    The core frequency is set to \SI{2.7}{\GHz}.}
\end{figure}

\clearpage
\begin{figure}[]
    \centering
    % width could be 1, 0.9, 0.8 instead
    \includegraphics[width=\columnwidth]{fig/rapl-update-intervals/MSR_PACKAGE_ENERGY_TIME_STATUS_2800000.pdf}
    \caption{Kernel density estimation of the time of measurement vs read energy value of the RAPL package counter of the first socket during for two constant power draw scenarios with and without filters enabled.
    The left plot uses the \textbf{rdtsc} function to get a timestamp for the RAPL measurement, the right uses the newly integrated timestamp in the \textbf{MSR\_PACKAGE\_ENERGY\_TIME\_STATUS} MSR.
    The core frequency is set to \SI{2.8}{\GHz}.}
\end{figure}

\begin{figure}[]
    \centering
    % width cloud be 0.54, 0.49, 0.44 instead
    \includegraphics[width=0.54\columnwidth]{fig/rapl-update-intervals/MSR_RAM_ENERGY_2800000.pdf}
    \caption{Kernel density estimation of the time of measurement via timestamp counter and the read energy value of the RAPL dram counter of the first socket during a constant power draw scenario.
    The core frequency is set to \SI{2.8}{\GHz}.}
\end{figure}

\clearpage
\begin{figure}[]
    \centering
    % width could be 1, 0.9, 0.8 instead
    \includegraphics[width=\columnwidth]{fig/rapl-update-intervals/MSR_PACKAGE_ENERGY_TIME_STATUS_2900000.pdf}
    \caption{Kernel density estimation of the time of measurement vs read energy value of the RAPL package counter of the first socket during for two constant power draw scenarios with and without filters enabled.
    The left plot uses the \textbf{rdtsc} function to get a timestamp for the RAPL measurement, the right uses the newly integrated timestamp in the \textbf{MSR\_PACKAGE\_ENERGY\_TIME\_STATUS} MSR.
    The core frequency is set to \SI{2.9}{\GHz}.}
\end{figure}

\begin{figure}[]
    \centering
    % width cloud be 0.54, 0.49, 0.44 instead
    \includegraphics[width=0.54\columnwidth]{fig/rapl-update-intervals/MSR_RAM_ENERGY_2900000.pdf}
    \caption{Kernel density estimation of the time of measurement via timestamp counter and the read energy value of the RAPL dram counter of the first socket during a constant power draw scenario.
    The core frequency is set to \SI{2.9}{\GHz}.}
\end{figure}

\clearpage
\begin{figure}[]
    \centering
    % width could be 1, 0.9, 0.8 instead
    \includegraphics[width=\columnwidth]{fig/rapl-update-intervals/MSR_PACKAGE_ENERGY_TIME_STATUS_3000000.pdf}
    \caption{Kernel density estimation of the time of measurement vs read energy value of the RAPL package counter of the first socket during for two constant power draw scenarios with and without filters enabled.
    The left plot uses the \textbf{rdtsc} function to get a timestamp for the RAPL measurement, the right uses the newly integrated timestamp in the \textbf{MSR\_PACKAGE\_ENERGY\_TIME\_STATUS} MSR.
    The core frequency is set to \SI{3.0}{\GHz}.}
\end{figure}

\begin{figure}[]
    \centering
    % width cloud be 0.54, 0.49, 0.44 instead
    \includegraphics[width=0.54\columnwidth]{fig/rapl-update-intervals/MSR_RAM_ENERGY_3000000.pdf}
    \caption{Kernel density estimation of the time of measurement via timestamp counter and the read energy value of the RAPL dram counter of the first socket during a constant power draw scenario.
    The core frequency is set to \SI{3.0}{\GHz}.}
\end{figure}

\clearpage
\begin{figure}[]
    \centering
    % width could be 1, 0.9, 0.8 instead
    \includegraphics[width=\columnwidth]{fig/rapl-update-intervals/MSR_PACKAGE_ENERGY_TIME_STATUS_3100000.pdf}
    \caption{Kernel density estimation of the time of measurement vs read energy value of the RAPL package counter of the first socket during for two constant power draw scenarios with and without filters enabled.
    The left plot uses the \textbf{rdtsc} function to get a timestamp for the RAPL measurement, the right uses the newly integrated timestamp in the \textbf{MSR\_PACKAGE\_ENERGY\_TIME\_STATUS} MSR.
    The core frequency is set to \SI{3.1}{\GHz}.}
\end{figure}

\begin{figure}[]
    \centering
    % width cloud be 0.54, 0.49, 0.44 instead
    \includegraphics[width=0.54\columnwidth]{fig/rapl-update-intervals/MSR_RAM_ENERGY_3100000.pdf}
    \caption{Kernel density estimation of the time of measurement via timestamp counter and the read energy value of the RAPL dram counter of the first socket during a constant power draw scenario.
    The core frequency is set to \SI{3.1}{\GHz}.}
\end{figure}

\clearpage
\begin{figure}[]
    \centering
    % width could be 1, 0.9, 0.8 instead
    \includegraphics[width=\columnwidth]{fig/rapl-update-intervals/MSR_PACKAGE_ENERGY_TIME_STATUS_3200000.pdf}
    \caption{Kernel density estimation of the time of measurement vs read energy value of the RAPL package counter of the first socket during for two constant power draw scenarios with and without filters enabled.
    The left plot uses the \textbf{rdtsc} function to get a timestamp for the RAPL measurement, the right uses the newly integrated timestamp in the \textbf{MSR\_PACKAGE\_ENERGY\_TIME\_STATUS} MSR.
    The core frequency is set to \SI{3.2}{\GHz}.}
\end{figure}

\begin{figure}[]
    \centering
    % width cloud be 0.54, 0.49, 0.44 instead
    \includegraphics[width=0.54\columnwidth]{fig/rapl-update-intervals/MSR_RAM_ENERGY_3200000.pdf}
    \caption{Kernel density estimation of the time of measurement via timestamp counter and the read energy value of the RAPL dram counter of the first socket during a constant power draw scenario.
    The core frequency is set to \SI{3.2}{\GHz}.}
\end{figure}

\clearpage
\begin{figure}[]
    \centering
    % width could be 1, 0.9, 0.8 instead
    \includegraphics[width=\columnwidth]{fig/rapl-update-intervals/MSR_PACKAGE_ENERGY_TIME_STATUS_3300000.pdf}
    \caption{Kernel density estimation of the time of measurement vs read energy value of the RAPL package counter of the first socket during for two constant power draw scenarios with and without filters enabled.
    The left plot uses the \textbf{rdtsc} function to get a timestamp for the RAPL measurement, the right uses the newly integrated timestamp in the \textbf{MSR\_PACKAGE\_ENERGY\_TIME\_STATUS} MSR.
    The core frequency is set to \SI{3.3}{\GHz}.}
\end{figure}

\begin{figure}[]
    \centering
    % width cloud be 0.54, 0.49, 0.44 instead
    \includegraphics[width=0.54\columnwidth]{fig/rapl-update-intervals/MSR_RAM_ENERGY_3300000.pdf}
    \caption{Kernel density estimation of the time of measurement via timestamp counter and the read energy value of the RAPL dram counter of the first socket during a constant power draw scenario.
    The core frequency is set to \SI{3.3}{\GHz}.}
\end{figure}

\clearpage
\begin{figure}[]
    \centering
    % width could be 1, 0.9, 0.8 instead
    \includegraphics[width=\columnwidth]{fig/rapl-update-intervals/MSR_PACKAGE_ENERGY_TIME_STATUS_3400000.pdf}
    \caption{Kernel density estimation of the time of measurement vs read energy value of the RAPL package counter of the first socket during for two constant power draw scenarios with and without filters enabled.
    The left plot uses the \textbf{rdtsc} function to get a timestamp for the RAPL measurement, the right uses the newly integrated timestamp in the \textbf{MSR\_PACKAGE\_ENERGY\_TIME\_STATUS} MSR.
    The core frequency is set to \SI{3.4}{\GHz}.}
\end{figure}

\begin{figure}[]
    \centering
    % width cloud be 0.54, 0.49, 0.44 instead
    \includegraphics[width=0.54\columnwidth]{fig/rapl-update-intervals/MSR_RAM_ENERGY_3400000.pdf}
    \caption{Kernel density estimation of the time of measurement via timestamp counter and the read energy value of the RAPL dram counter of the first socket during a constant power draw scenario.
    The core frequency is set to \SI{3.4}{\GHz}.}
\end{figure}

\clearpage
\begin{figure}[]
    \centering
    % width could be 1, 0.9, 0.8 instead
    \includegraphics[width=\columnwidth]{fig/rapl-update-intervals/MSR_PACKAGE_ENERGY_TIME_STATUS_3500000.pdf}
    \caption{Kernel density estimation of the time of measurement vs read energy value of the RAPL package counter of the first socket during for two constant power draw scenarios with and without filters enabled.
    The left plot uses the \textbf{rdtsc} function to get a timestamp for the RAPL measurement, the right uses the newly integrated timestamp in the \textbf{MSR\_PACKAGE\_ENERGY\_TIME\_STATUS} MSR.
    The core frequency is set to \SI{3.5}{\GHz}.}
\end{figure}

\begin{figure}[]
    \centering
    % width cloud be 0.54, 0.49, 0.44 instead
    \includegraphics[width=0.54\columnwidth]{fig/rapl-update-intervals/MSR_RAM_ENERGY_3500000.pdf}
    \caption{Kernel density estimation of the time of measurement via timestamp counter and the read energy value of the RAPL dram counter of the first socket during a constant power draw scenario.
    The core frequency is set to \SI{3.5}{\GHz}.}
\end{figure}

\clearpage
\begin{figure}[]
    \centering
    % width could be 1, 0.9, 0.8 instead
    \includegraphics[width=\columnwidth]{fig/rapl-update-intervals/MSR_PACKAGE_ENERGY_TIME_STATUS_3600000.pdf}
    \caption{Kernel density estimation of the time of measurement vs read energy value of the RAPL package counter of the first socket during for two constant power draw scenarios with and without filters enabled.
    The left plot uses the \textbf{rdtsc} function to get a timestamp for the RAPL measurement, the right uses the newly integrated timestamp in the \textbf{MSR\_PACKAGE\_ENERGY\_TIME\_STATUS} MSR.
    The core frequency is set to \SI{3.6}{\GHz}.}
\end{figure}

\begin{figure}[]
    \centering
    % width cloud be 0.54, 0.49, 0.44 instead
    \includegraphics[width=0.54\columnwidth]{fig/rapl-update-intervals/MSR_RAM_ENERGY_3600000.pdf}
    \caption{Kernel density estimation of the time of measurement via timestamp counter and the read energy value of the RAPL dram counter of the first socket during a constant power draw scenario.
    The core frequency is set to \SI{3.6}{\GHz}.}
\end{figure}

\clearpage
\begin{figure}[]
    \centering
    % width could be 1, 0.9, 0.8 instead
    \includegraphics[width=\columnwidth]{fig/rapl-update-intervals/MSR_PACKAGE_ENERGY_TIME_STATUS_3700000.pdf}
    \caption{Kernel density estimation of the time of measurement vs read energy value of the RAPL package counter of the first socket during for two constant power draw scenarios with and without filters enabled.
    The left plot uses the \textbf{rdtsc} function to get a timestamp for the RAPL measurement, the right uses the newly integrated timestamp in the \textbf{MSR\_PACKAGE\_ENERGY\_TIME\_STATUS} MSR.
    The core frequency is set to \SI{3.7}{\GHz}.}
\end{figure}

\begin{figure}[]
    \centering
    % width cloud be 0.54, 0.49, 0.44 instead
    \includegraphics[width=0.54\columnwidth]{fig/rapl-update-intervals/MSR_RAM_ENERGY_3700000.pdf}
    \caption{Kernel density estimation of the time of measurement via timestamp counter and the read energy value of the RAPL dram counter of the first socket during a constant power draw scenario.
    The core frequency is set to \SI{3.7}{\GHz}.}
\end{figure}

\clearpage
\begin{figure}[]
    \centering
    % width could be 1, 0.9, 0.8 instead
    \includegraphics[width=\columnwidth]{fig/rapl-update-intervals/MSR_PACKAGE_ENERGY_TIME_STATUS_3800000.pdf}
    \caption{Kernel density estimation of the time of measurement vs read energy value of the RAPL package counter of the first socket during for two constant power draw scenarios with and without filters enabled.
    The left plot uses the \textbf{rdtsc} function to get a timestamp for the RAPL measurement, the right uses the newly integrated timestamp in the \textbf{MSR\_PACKAGE\_ENERGY\_TIME\_STATUS} MSR.
    The core frequency is set to \SI{3.8}{\GHz}.}
\end{figure}

\begin{figure}[]
    \centering
    % width cloud be 0.54, 0.49, 0.44 instead
    \includegraphics[width=0.54\columnwidth]{fig/rapl-update-intervals/MSR_RAM_ENERGY_3800000.pdf}
    \caption{Kernel density estimation of the time of measurement via timestamp counter and the read energy value of the RAPL dram counter of the first socket during a constant power draw scenario.
    The core frequency is set to \SI{3.8}{\GHz}.}
\end{figure}


\chapter{Appendix P-State Latencies}
\label{app:pstate_latencies_scatter_complete}
This section includes the scatter plots for all measurement points from the ftalat benchmark as described in~\secref{pstate_latencies}.

\begin{figure}[]
    \centering
    \includegraphics[width=\columnwidth]{fig/ftalat/ftalat_scatter_wait_transition_latency_hati_source_0.8.pdf}
    \caption{Dependance of the time between the detection events of transitions from \SI{0.8}{\GHz} to \SI{0.9}{}, \SI{1.0}{}, ..., \SI{2.9}{\GHz}. For better lines for bins of size \SI{1625}{\us} and \SI{200}{\us} have been included for the x and y-axis respectively.}
\end{figure}
\begin{figure}[]
    \centering
    \includegraphics[width=\columnwidth]{fig/ftalat/ftalat_scatter_wait_transition_latency_hati_source_0.9.pdf}
    \caption{Dependance of the time between the detection events of transitions from \SI{0.9}{\GHz} to \SI{0.8}{}, \SI{1.0}{}, ..., \SI{2.9}{\GHz}. For better lines for bins of size \SI{1625}{\us} and \SI{200}{\us} have been included for the x and y-axis respectively.}
\end{figure}
\begin{figure}[]
    \centering
    \includegraphics[width=\columnwidth]{fig/ftalat/ftalat_scatter_wait_transition_latency_hati_source_1.0.pdf}
    \caption{Dependance of the time between the detection events of transitions from \SI{1.0}{\GHz} to \SI{0.8}{}, \SI{0.9}{}, ..., \SI{2.9}{\GHz}. For better lines for bins of size \SI{1625}{\us} and \SI{200}{\us} have been included for the x and y-axis respectively.}
\end{figure}
\begin{figure}[]
    \centering
    \includegraphics[width=\columnwidth]{fig/ftalat/ftalat_scatter_wait_transition_latency_hati_source_1.1.pdf}
    \caption{Dependance of the time between the detection events of transitions from \SI{1.1}{\GHz} to \SI{0.8}{}, \SI{0.9}{}, ..., \SI{2.9}{\GHz}. For better lines for bins of size \SI{1625}{\us} and \SI{200}{\us} have been included for the x and y-axis respectively.}
\end{figure}
\begin{figure}[]
    \centering
    \includegraphics[width=\columnwidth]{fig/ftalat/ftalat_scatter_wait_transition_latency_hati_source_1.2.pdf}
    \caption{Dependance of the time between the detection events of transitions from \SI{1.2}{\GHz} to \SI{0.8}{}, \SI{0.9}{}, ..., \SI{2.9}{\GHz}. For better lines for bins of size \SI{1625}{\us} and \SI{200}{\us} have been included for the x and y-axis respectively.}
\end{figure}
\begin{figure}[]
    \centering
    \includegraphics[width=\columnwidth]{fig/ftalat/ftalat_scatter_wait_transition_latency_hati_source_1.3.pdf}
    \caption{Dependance of the time between the detection events of transitions from \SI{1.3}{\GHz} to \SI{0.8}{}, \SI{0.9}{}, ..., \SI{2.9}{\GHz}. For better lines for bins of size \SI{1625}{\us} and \SI{200}{\us} have been included for the x and y-axis respectively.}
\end{figure}
\begin{figure}[]
    \centering
    \includegraphics[width=\columnwidth]{fig/ftalat/ftalat_scatter_wait_transition_latency_hati_source_1.4.pdf}
    \caption{Dependance of the time between the detection events of transitions from \SI{1.4}{\GHz} to \SI{0.8}{}, \SI{0.9}{}, ..., \SI{2.9}{\GHz}. For better lines for bins of size \SI{1625}{\us} and \SI{200}{\us} have been included for the x and y-axis respectively.}
\end{figure}
\begin{figure}[]
    \centering
    \includegraphics[width=\columnwidth]{fig/ftalat/ftalat_scatter_wait_transition_latency_hati_source_1.5.pdf}
    \caption{Dependance of the time between the detection events of transitions from \SI{1.5}{\GHz} to \SI{0.8}{}, \SI{0.9}{}, ..., \SI{2.9}{\GHz}. For better lines for bins of size \SI{1625}{\us} and \SI{200}{\us} have been included for the x and y-axis respectively.}
\end{figure}
\begin{figure}[]
    \centering
    \includegraphics[width=\columnwidth]{fig/ftalat/ftalat_scatter_wait_transition_latency_hati_source_1.6.pdf}
    \caption{Dependance of the time between the detection events of transitions from \SI{1.6}{\GHz} to \SI{0.8}{}, \SI{0.9}{}, ..., \SI{2.9}{\GHz}. For better lines for bins of size \SI{1625}{\us} and \SI{200}{\us} have been included for the x and y-axis respectively.}
\end{figure}
\begin{figure}[]
    \centering
    \includegraphics[width=\columnwidth]{fig/ftalat/ftalat_scatter_wait_transition_latency_hati_source_1.7.pdf}
    \caption{Dependance of the time between the detection events of transitions from \SI{1.7}{\GHz} to \SI{0.8}{}, \SI{0.9}{}, ..., \SI{2.9}{\GHz}. For better lines for bins of size \SI{1625}{\us} and \SI{200}{\us} have been included for the x and y-axis respectively.}
\end{figure}
\begin{figure}[]
    \centering
    \includegraphics[width=\columnwidth]{fig/ftalat/ftalat_scatter_wait_transition_latency_hati_source_1.8.pdf}
    \caption{Dependance of the time between the detection events of transitions from \SI{1.8}{\GHz} to \SI{0.8}{}, \SI{0.9}{}, ..., \SI{2.9}{\GHz}. For better lines for bins of size \SI{1625}{\us} and \SI{200}{\us} have been included for the x and y-axis respectively.}
\end{figure}
\begin{figure}[]
    \centering
    \includegraphics[width=\columnwidth]{fig/ftalat/ftalat_scatter_wait_transition_latency_hati_source_1.9.pdf}
    \caption{Dependance of the time between the detection events of transitions from \SI{1.9}{\GHz} to \SI{0.8}{}, \SI{0.9}{}, ..., \SI{2.9}{\GHz}. For better lines for bins of size \SI{1625}{\us} and \SI{200}{\us} have been included for the x and y-axis respectively.}
\end{figure}
\begin{figure}[]
    \centering
    \includegraphics[width=\columnwidth]{fig/ftalat/ftalat_scatter_wait_transition_latency_hati_source_2.0.pdf}
    \caption{Dependance of the time between the detection events of transitions from \SI{2.0}{\GHz} to \SI{0.8}{}, \SI{0.9}{}, ..., \SI{2.9}{\GHz}. For better lines for bins of size \SI{1625}{\us} and \SI{200}{\us} have been included for the x and y-axis respectively.}
\end{figure}
\begin{figure}[]
    \centering
    \includegraphics[width=\columnwidth]{fig/ftalat/ftalat_scatter_wait_transition_latency_hati_source_2.1.pdf}
    \caption{Dependance of the time between the detection events of transitions from \SI{2.1}{\GHz} to \SI{0.8}{}, \SI{0.9}{}, ..., \SI{2.9}{\GHz}. For better lines for bins of size \SI{1625}{\us} and \SI{200}{\us} have been included for the x and y-axis respectively.}
\end{figure}
\begin{figure}[]
    \centering
    \includegraphics[width=\columnwidth]{fig/ftalat/ftalat_scatter_wait_transition_latency_hati_source_2.2.pdf}
    \caption{Dependance of the time between the detection events of transitions from \SI{2.2}{\GHz} to \SI{0.8}{}, \SI{0.9}{}, ..., \SI{2.9}{\GHz}. For better lines for bins of size \SI{1625}{\us} and \SI{200}{\us} have been included for the x and y-axis respectively.}
\end{figure}
\begin{figure}[]
    \centering
    \includegraphics[width=\columnwidth]{fig/ftalat/ftalat_scatter_wait_transition_latency_hati_source_2.3.pdf}
    \caption{Dependance of the time between the detection events of transitions from \SI{2.3}{\GHz} to \SI{0.8}{}, \SI{0.9}{}, ..., \SI{2.9}{\GHz}. For better lines for bins of size \SI{1625}{\us} and \SI{200}{\us} have been included for the x and y-axis respectively.}
\end{figure}
\begin{figure}[]
    \centering
    \includegraphics[width=\columnwidth]{fig/ftalat/ftalat_scatter_wait_transition_latency_hati_source_2.4.pdf}
    \caption{Dependance of the time between the detection events of transitions from \SI{2.4}{\GHz} to \SI{0.8}{}, \SI{0.9}{}, ..., \SI{2.9}{\GHz}. For better lines for bins of size \SI{1625}{\us} and \SI{200}{\us} have been included for the x and y-axis respectively.}
\end{figure}
\begin{figure}[]
    \centering
    \includegraphics[width=\columnwidth]{fig/ftalat/ftalat_scatter_wait_transition_latency_hati_source_2.5.pdf}
    \caption{Dependance of the time between the detection events of transitions from \SI{2.5}{\GHz} to \SI{0.8}{}, \SI{0.9}{}, ..., \SI{2.9}{\GHz}. For better lines for bins of size \SI{1625}{\us} and \SI{200}{\us} have been included for the x and y-axis respectively.}
\end{figure}
\begin{figure}[]
    \centering
    \includegraphics[width=\columnwidth]{fig/ftalat/ftalat_scatter_wait_transition_latency_hati_source_2.6.pdf}
    \caption{Dependance of the time between the detection events of transitions from \SI{2.6}{\GHz} to \SI{0.8}{}, \SI{0.9}{}, ..., \SI{2.9}{\GHz}. For better lines for bins of size \SI{1625}{\us} and \SI{200}{\us} have been included for the x and y-axis respectively.}
\end{figure}
\begin{figure}[]
    \centering
    \includegraphics[width=\columnwidth]{fig/ftalat/ftalat_scatter_wait_transition_latency_hati_source_2.7.pdf}
    \caption{Dependance of the time between the detection events of transitions from \SI{2.7}{\GHz} to \SI{0.8}{}, \SI{0.9}{}, ..., \SI{2.9}{\GHz}. For better lines for bins of size \SI{1625}{\us} and \SI{200}{\us} have been included for the x and y-axis respectively.}
\end{figure}
\begin{figure}[]
    \centering
    \includegraphics[width=\columnwidth]{fig/ftalat/ftalat_scatter_wait_transition_latency_hati_source_2.8.pdf}
    \caption{Dependance of the time between the detection events of transitions from \SI{2.8}{\GHz} to \SI{0.8}{}, \SI{0.9}{}, ..., \SI{2.9}{\GHz}. For better lines for bins of size \SI{1625}{\us} and \SI{200}{\us} have been included for the x and y-axis respectively.}
\end{figure}
\begin{figure}[]
    \centering
    \includegraphics[width=\columnwidth]{fig/ftalat/ftalat_scatter_wait_transition_latency_hati_source_2.9.pdf}
    \caption{Dependance of the time between the detection events of transitions from \SI{2.9}{\GHz} to \SI{0.8}{}, \SI{0.9}{}, ..., \SI{2.9}{\GHz}. For better lines for bins of size \SI{1625}{\us} and \SI{200}{\us} have been included for the x and y-axis respectively.}
\end{figure}