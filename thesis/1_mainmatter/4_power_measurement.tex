\chapter{Power Measurement}

\section{Validating RAPL Accuracy}
Starting with the Haswell microarchitecture Intel integrated current measurement into their processor to support power limiting~\cite{Hackenberg_2015_Haswell}.
The associated power metrics for different zones can be measured using the RAPL interface via the linux kernel~\cite{powercap_kernel_doc}.
Schöne et al.~\cite{Schoene_2024_Alder_Lake} tried to validate that RAPL counters on the Alder Lake architecture rely on current measurements, however they found that they are likely using a model.
This is relevant to Sapphire Rapids, since they share the same core microarchitecture Golden Cove.

I validate the accuracy of these counters against an external measurement using roco2 synthetic workload generator.
This software required some patches to run on the current generation of processors.
They are documented in the forked GitHub repository~\footnote{\url{https://github.com/marenz2569/roco2/tree/marenz.hati-config}}.

Each of the displayed workload in~\figref{validate-rapl} were run for \SI{60}{\second} on the cross product of following settings:
\begin{itemize}
    \item Core frequency set to \SI{800}{\MHz}, \SI{1400}{\MHz}, \SI{2000}{\MHz} and \SI{3800}{\MHz} (turbo).
    \item Four settings which represent the execution of the workload on an increasing number of quadrants on the first socket.
    The number of cores used are: \SI{14}{}, \SI{28}{}, \SI{42}{} and \SI{56}{}.
\end{itemize}
The test matrix further included a full idle for each C-state setting (POLL, C1, C1E, C6).
Hyperthreading is disabled and the governor set to \texttt{performance}.
The power metrics were measured through the FIRESTARTER measurement inferface.
This polls the RAPL metrics every \SI{10}{\ms} and discovered the first and last \SI{5}{\second} of each measurement duration.
The external reference measurent of the PDU is exposed via the metricq~\cite{Ilsche_2019_MetricQ} interface of FIRESTARTER~\footnote{\url{https://github.com/marenz2569/firestarter-metric-metricq}} and reports \SI{1}{Sa\per\second}.
The average power draw of the PDU and the RAPL counters is plotted in~\figref{validate-rapl}.
A strong correlation via quadratic fit can be observed, indication that RAPL metrics are being measured.
All measurment points are inside the \SI{1}{\percent} tolerance of the PDU.

\begin{figure}[]
    \centering
    \includegraphics[width=0.8\columnwidth]{fig/rapl-accuracy/rapl-accuracy.pdf}
    \caption{\label{fig:validate-rapl}The roco2 microbenchmark is executed on a varying number of cores with different frequencies.
    The RAPL measurement can be mapped with a quadratic fit to the external reference measurement.}
\end{figure}

\section{RAPL Filters}

Energy measurement with high accuracy can be used to measure the energy difference between different executed instructions and their data.
This has been demonstrated in great detail~\cite{Lucas_2016_AluPower,Schoene_2024_Alder_Lake,Schoene_2021_Zen2}.
However it can also be used to perform side channel attacks and retrieve keys from priviledged programs~\cite{Lipp_2021_Platypus}.
While limiting access to the RAPL counters is possible from an operating system perspective, Intel also provides the option to enable filters \textbf{IA32\_MISC\_PACKAGE\_CTLS} on the energy measurement which cannot be deactived until a reboot is performed~\cite[Vol. 4 Table 2-52]{intel_combined_software_developer_manual}.
The third generation of Xeon Scalable processors (Ice Lake) introduced the \textbf{MSR\_PACKAGE\_ENERGY\_TIME\_STATUS} MSR which also contains an accurate time point when the register was last updated.
\todoms{This addresses a problem which has been highlighted by et al.~\cite{} to allow for accurate energy accounting.}

I extend the measurement for RAPL granularity propsed by Schöne et al.~\cite{Schoene_2024_Alder_Lake}.
All possible RAPL counters are polled repeatidly for \SI{10}{\s} on CPU 0 via their associated MSR.
CPU 1 runs FIRESTARTER to generate a constant power consumption on the processor.
This measurement is repeated for all core frequencies, with and without filters enabled.

The counters \textbf{MSR\_PCKG\_ENERGY}, \textbf{MSR\_PACKAGE\_ENERGY\_TIME\_STATUS} and \textbf{MSR\_RAM\_ENERGY} contain valid values.
I plot the package counters in~\figref{rapl-update-intervals-package} and dram counters in%~\figreg{rapl-update-intervals-dram}.
Interestingly for \SI{800}{\MHz} core frequency no update of the value from the \textbf{MSR\_PACKAGE\_ENERGY\_TIME\_STATUS} metric is observed when the filter and all cstates are actitvated.

Data center and HPC administrators require accuracte energy accounting of larger code paths or accross multiple executions.
Software developers require it for fast runtime optimization of energy efficiency.
An increased accuracy of this measurement is possible with the use of the internal timestamp in the \textbf{MSR\_PACKAGE\_ENERGY\_TIME\_STATUS} register.
For use cases with security critical data and algorithms, I recommend to activate the filter increasing the barrier against leaks even if parts of the system are compromised.

\begin{figure}[]
    \centering
    \includegraphics[width=\columnwidth]{fig/rapl-update-intervals/MSR_PACKAGE_ENERGY_TIME_STATUS_2000000.pdf}
    \caption{\label{fig:rapl-update-intervals-package}Kernel density estimation of the time of measurement vs read energy value of the RAPL package counter of the first socket during a constant power draw scenario.
    The time value is displayed for both the time from the timestamp instruction and the reported time the RAPL model specific register.}
\end{figure}