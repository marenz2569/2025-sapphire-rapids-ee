\chapter{Power Management}
``It will also be interesting to look at Sapphire Rapids, which also implements Golden Cove cores, including an analysis of user space idle states, and AVX frequencies.''

\section{Infuence of Tiles}
\todoms{Where is the PMU?}

\section{Shared Frequency Domains}

\section{AVX Frequencies}

Starting with the Haswell Microarchitecture, Intel introduced the concept of AVX frequencies.
This allow the AVX2 instruction to use a lower base and different per core turbo frequencies.~\cite{Hackenberg_2015_Haswell}
The concept has been extended with a new class for AVX512 instructions with the Skylake Server Microarchitecture.~\cite[Sec. 2.6.3]{Intel_Optimization_Reference_Manual_050}
With the introduction of Ice Lake these classes are not only determined by the instruction set or register size, but a more realistic model of actual power consumption by also classifing instruction into ``Light'' and ``Heavy'' classes by their power consumption.~\cite{papazian_new_2020}
With the addition of the AMX instruction set Intel split up these classes further ranging from ``Ultra-Light'' to ``Heavy''.
The used instruction set together with the power consumption are mapped to four license levels which determine the opportunistic AVX turbo frequencies based on the number of active cores.
The mapping from instruction set and power usage are displayed in~\tabref{avx-classes}.~\cite{ServeTheHome_Emerald_Rapids_2023}
In \figref{p0n-frequencies} we are able to extracte the per core opportunistic turbo frequencies per license on our testsystem using a modified version of ``intel-speed-select''.
\todoms{Reference to the test system}

\begin{table}[t]
	\centering
	\caption{\label{tab:avx-classes}AVX frequency classes based on published slides from Intel.~\cite{ServeTheHome_Emerald_Rapids_2023}}
	\begin{tabular}{|l|c|c|c|c|}
        \hline
        \diagbox[height=5em]{Instruction\\Class}{\\$C_{dyn}$ Class} & 0 & 1 & 2 & 3 \\
        \hline
        SSE & 128 Light & 128 Heavy & & \\
        AVX2 & 256 Light & 256 Moderate & 256 Heavy & \\
        AVX512 & 512 Ultra-Light & 512 Light & 512 Moderate & 512 Heavy \\
        AMX & & AMX Light & AMX Moderate & AMX Heavy \\
        \hline \hline
        Turbo Frequency & SSE & AVX2 & AVX512 & AMX \\ 
        \hline
	\end{tabular}
\end{table}

In~\cite{Downs_2020_AVX_Downclocking}, Downs measured the influence of AVX throttling on Ice Lake by infering the license level based on the resulting frequency of the ``avx-turbo'' microarchitectural benchmark.
However they were only able to trigger two out of the three license level, leaving open the question on how instructions are mapped to ``Light'' and ``Heavy'' classes.
Laukemann et al.~\cite{laukemann_microarchitectural_2024} showed the same for Sapphire Rapids using ``likwid-bench'' with the ``peakflops\_avx512\_fma'' and ``peakflops\_avx\_fma'' microarchitectural benchmark.
They we also only able to show two out of the four license levels.

\todoms{Show the license levels on SPR barnard (same processor as laukemann)}

\begin{figure}[]
    \centering
    \includegraphics[width=0.8\columnwidth]{fig/avx-frequency-license-bands.pdf}
    \caption{\label{fig:p0n-frequencies}Opportunistic AVX turbo frequencies of the test system extracted using Intel Speed Select Technology.
    The number of active cores are split into eight buckets. The license and the bucket decides the resulting turbo frequency.
    In the first bucket the SSE and AVX2 license bands share the same frequency.}
\end{figure}

\todoms{Reading the cdynn classes from the PMU (per core)}

\section{Uncore Frequency Scaling}

\section{Idle State Latencies}
\todoms{Both user and normal C states}