\chapter{Introduction}
\label{sec:introduction}

While computational throughput is an important metric in the evaluation of processors, many applications do not reach peak throughput being limited by power, memory througput or latency.
With the increased growth of compute intensive workloads, even small energy efficiency improvements can significantly reduce operational costs of \ac{HPC} clusters.
These improvements can, however increase the total runtime of workloads and latencies that occur in critical regions.

Many of the mechanisms influencing energy efficiency are highly configurable, but also significantly impact power usage, throughput and latency.
P-states allow setting specific frequency ranges that are used by internal control loops to set a frequency that is optimal for an executed workload.
C-states aim to increase energy effiency during idle periods and can be invoked by both the kernel and user-space applications.
Turbo frequencies improve the performance of the executed workload while the processor is not power limited.
Power limiting ensures, that the processor stays inside specified thermal and electrical limits.
Power, frequency and memory allocation mechanisms allow to efficiently run inbalanced or multiple workloads on the same processor.
This thesis takes a deep look at Intels continued improvements of these mechanisms for the Sapphire Rapids processor generation.
Rather then looking at application specific energy efficiency improvements, I provide insight into the complex interactions between hardware control mechanisms and the operation system that help to optimize energy efficiency.

In~\secref{background}, I discuss exsisting mechanisms to optimize energy efficiency and the evaluation thereof.
\secref{arch} describes the architecture of the Sapphire Rapids processor, introduces the test system and shows benchmarks of the core and uncore microarchitecture.
In~\secref{power_measurement}, I validate the accuracy of the power measurement which is used to facilitate some power management tasks, i.e. frequency throttling.
\secref{power_management} covers key metrics of exsisting power management mechanisms, i.e. turbo frequencies, frequency scaling and idle state latencies and shows changes in comparison with previous processor generations.
I conclude with a summary and outlook in~\secref{summary}.