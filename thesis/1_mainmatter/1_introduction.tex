\chapter{Introduction}
\label{sec:introduction}

While computational throughput is an important metric in the evaluation of processors, many applications do not reach peak throughput and are limited by power, memory througput or latency.
With the increased growth of compute intensive workloads, even small energy efficiency improvements can significantly reduce operational costs of HPC clusters.
These improvements can however increase the total runtime of workloads and latencies that occur in critical regions.

Many mechanism that influence energy efficiency are highly configurable, but also significantly influce power usage, throughput and latency.
P-states allow setting specific frequency ranges that are used by internal control loops to set a frequency that is optimal for a executed workload.
C-states aid to increase energy effiency during idle periods and can be invoked by the kernel and user-space applications.
Turbo frequencies improve the performance of the processors execution while it is not power limited.
Power limiting ensure that the processor stays inside specified thermal and electrical limits.
Power, frequency and memory allocation mechanism allow to efficiently run inbalanced or multiple workloads on the same processor.
This thesis looks at the continued improvements of the Intel Sapphire Rapids processor generation.
Rather then looking at application specific energy efficiency improvements, I provide insight in the complex interactions between hardware control mechanisms and the operation system that help to optimize energy efficiency.

In~\secref{background}, I discuss exsisting mechanisms to optimize energy efficiency and the evaluation thereof.
\secref{arch} describes the architecture of the Sapphire Rapids processor, introduces the test system and shows benchmarks of the core and uncore microarchitecture.
In~\secref{power_measurement}, I validate the accuracy of the power measurement which is used to facilitate some power management tasks, i.e. frequency throttling.
\secref{power_management} covers key metrics of exsisting power management mechanisms, i.e. turbo frequencies, frequency scaling and idle state latencies, and shows changes compared to previous processor generations.
I conclude with a summary and outlook in~\secref{summary}.